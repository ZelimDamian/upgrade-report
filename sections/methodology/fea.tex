\section{Finite Element Method soft Tissue Simulation}\label{methodology-fea}

\subsection{FEA concepts}

\subsection{Total lagrangian explicit dynamic FE}

  Total lagrangian explicit dynamic (TLED) is a variation on the Lagrangian methods to finite element formulations. It is intially proposed in \cite{Miller2007} as a means for computer based finite element analysis. The alternative approach is known as the Updated lagrangian explicit dynamic approach is based on using the previously calculated configuration of the deformed body in order to calculate the stresses occuring in the finite elements. In contast to this technique TLED uses the original reference configuration to perform the calculations. TLED has the advantage of allowing precomputations of a large chunk of data as compared to the approach using the updated lagrangian.

  \subsubsection{NiftySim TLED FEM software}

  \citet{Johnsen2014}

\subsection{CPU based implementation}

  \subsubsection{Traditional FEA simulation}

  Traditionally FE analyses are performed on a particular setup with fixed parameters. Time frame is one the most important aspects of the simulation. To perform an analysis a starting time and a finishing time are normally chosen and the simulation is commenced. The natural and essential external loads are varied throughout the timeframe according to well defined laws and functions.

  The developed BrithView TLED simulation system is capable of performing this type of a simulation. The dedicated component class called \textit{FixedTimeSimulationComponent} can be used to perform a fixed-time simulation.

  The advantage of such type of simulations is apparent when a highly controlled analysis environment is required. When testing the stability and strength of mechanical structures such simulations allow greater control and repeatability of the analyses.

  However, there are cases when the analysis is required to flow continuously throughout the simulation. For such cases we have developed an alternative simulation framework described in section \ref{methodology-fea-interactive}.

  \subsubsection{Interactive real-time simulation}\label{methodology-fea-interactive}

  In contrast to the previously described type of FE simulation, an interactive simulation is not limited to a fixed timeframe.
