This chapter outlines future work to be performed as part of the project, split into different areas of focus.

\section{Experiments}

The main target of this research is establishing all seven cardinal movements of human labour in a simulated environment without imposed trajectories. Purely physics based simulation is to be used to accomplish this, as opposed to the hybrid method described in the previous section.

Unfortunately we do not possess any quantitative data on the trajectory of fetal descent that can be used to meaningfully compare with the observed trajectory during our simulation.

\subsection{Bony structures with a basic pelvic floor model}
A scenario where the fetal skull is descending through the maternal pelvis with a simple pelvic floor model attached. This scenario is a potential starting point for simulation experiments. This experiment will be directed at validating the simulation system. The work presented in \cite{parente2014} !!! represents a very simillar scenario, although more simplified. We beleive that having a realistic skull model will improve fidelity of the observed experimental values.

\section{Volumetric mesh generation}\label{future-meshgen}

Mesh generation is designated as one of the higher priorty tasks to accomplish. Having acquired or generated high quality tetrahedral or hexahedral meshes is a crucial requirement in being able to perform the research experiments.

\section{Improved contact model}\label{future-meshgen}

Mesh generation is designated as one of the higher priorty tasks to accomplish. Having acquired or generated high quality tetrahedral or hexahedral meshes is a crucial requirement in being able to perform the research experiments.
