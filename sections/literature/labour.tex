\section{Human labour}

\subsection{Cardinal movements}

  The cardinal movements of human labour are the main focus of this research project. It is important to find the current state of the research body related to this topic.

  \subsubsection{Qualitative assesments}

  \citep{abitbol1993} is an example of a scientific clinical observeration which postulated a qualitative assesment of the reasons why the cardinal movements occurr in normal human labour. The difference between how non-human and human anthropoids go through labour is emphasized by the authors. Human obstetrics are stated to be different due to the locomotion adaptations and fetal encephalization. Upright walking and greater fetal head diameters are named as the reasons behind the more complicated human labour. However, as we know the human labour is a successful means of partuition none the less. The adaptations that were

  The most important reasons for the successful delivery are stated as follows:

  \begin{enumerate}

    \item Spheroidicity of the presenting part of the fetal head
    \item Mobility of the head and chest
    \item Fetal skull molding which adapts the head diameters to fit the maternal pelvis

  \end{enumerate}

  The first and second items in the list are very closely related to the phenomen of cardinal movements. The authors state that the spherical shape of the presenting side of the head allows it to \"roll\" in the pelvis. This rolling motion can also be described as the flexion, extension and internal rotation. Which again advocates the point that the cardinal movements are the key in successful normal human labour.

  \paragraph{Discussion} Although a good qualitative explanation is provided for the way in which successful human labour occurrs, no details are given for the actual mechanism. The paper focuses on stating the causes of the complicated nature of human labour as compared to non-humans and provides the details of how the complications are overcome. However, no quantitative data is avaiable for the mechanics based description normal human labour process.
