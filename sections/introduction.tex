Computer based simulations find numerous applications in medicine. Such applications include training of medical personell, diagnosing patients based on digital data and scientific research to gain better understanding of physiological phenomena. One of the most challenging applications of computer based simulation in medicine is bio-mechanical simuations. The underlying principles are very complex and thus require sophisticated theoretical formulations and considerable software development efforts.



Computer simulations provide a good tool for representing real world phenomena, but they are only capable of representing the simulated objects to a certain degree of approximation. Better approximations are predominantly much more expensive in terms of computational power. With the increased processing power of modern computers, it is possible to perform simuations with a higher degree of fidelity. However, even the most performant machines can struggle with certain types of high-cost simulations. In such cases, we have to utilize the underlying hardware to the highest degree possible. This can be achieved by a number optimization techniques. One of the most effective techniques is using parallel processing in order to speed up the computation.

It is desired to achieve the highest fidelity of the simulaitons with the as little latency as possible. Therefore, performance optimizations and GPGPU utilization for Finite Element Analysis is one of the main focuses of this thesis. Several available application programming interfaces (API's) will be overviewed as the candidates for the implementation. It is then shown how the chosen API is used to achieve highly efficient implementation of FEA for soft-tissue simulation.

Another important aspect of creating a computer based simulation of childbirth is acquiring realistic 3D models of the underlying physiological structures. Namely, the fetal body and maternal lower body geometries are required. The possible ways of constructing the required meshes will be covered in this thesis. The approach is not yet decided on.

The remainder of this report is arranged as follows. In chapter \ref{chap-literature} a literature review is conducted  presenting the body of already existing relevant research. Chapter \ref{chap-methodology} narrates the .(a paper covering work performed in benchmarking reverse engineering is also included in Appendix \ref{redbm-paper}, with details of the benchmark process and tooling in Appendix \ref{work-benchmark-process-app}). Future work is identified in chapter \ref{chap-future} including a work plan and Gantt chart. Finally chapter \ref{chap-thesis} presents a proposed thesis structure for the final write-up.
